\documentclass{subfiles}

\begin{document}
\begin{table}[H]
    \centering
    \begin{tabular}{|ccc|}
        \textbf{Folienbereich} & \textbf{Aufgaben} & \textbf{Zeit} \\
        \hline\hline
        \pgfmathparse{\Kapitelseiten[0]}\pgfmathresult & \pgfmathparse{\Aufgaben[0]}\pgfmathresult & \pgfmathparse{\Zeiten[0]}\pgfmathresult
    \end{tabular}
\end{table}
In diesem Kapitel beschäftigen wir uns mit der \emph{Software}, mit welcher \LaTeX{}-Code formuliert werden kann. Hier bieten wir primär \emph{TeXStudio} oder \emph{Visual Studio Code} als lokale Lösungen und \emph{Overleaf} als Cloudlösung an. Wir wollen jedoch auf weitere Software wie \emph{texifier} (ehem. TeXPad, macOS) oder einfache Texteditoren wie \emph{Notepad++} (Windows) oder \emph{VIM}, \emph{Nano} bzw. \emph{Emacs} (Linux) hinweisen. \\

Bei der Kompilierung beschäftigen wir uns hauptsächlich mit \emph{pdflatex}. \\

Für den Grundaufbau eines \LaTeX{} Dokumentes schauen wir uns die eine grundlagende Struktur an:
\lstinputlisting[language=TeX]{../Inhalte/Konzepte/Beispiel.tex}
\end{document}