\documentclass{subfiles}

\begin{document}
    \begin{table}[H]
        \centering
        \begin{tabular}{|ccc|}
            \textbf{Folienbereich} & \textbf{Aufgaben} & \textbf{Zeit} \\
            \hline\hline
            \pgfmathparse{\Kapitelseiten[9]}\pgfmathresult & \pgfmathparse{\Aufgaben[9]}\pgfmathresult & \pgfmathparse{\Zeiten[9]}\pgfmathresult
        \end{tabular}
    \end{table}

    Für eine optimale Abgabe eines Dokumentes ist das Titelbild kaum wegzudenken. Den \LaTeX{} eigenen \emph{Titelstil} werden wir zusammen mit der Definition von \enquote{title}, \enquote{author} und \enquote{date} kennenlernen. \\

    Darauf aufbauend betrachten wir die \emph{Konfiguration} des \LaTeX{} Dokumentes über unsere Standardformatierung hinaus und werfen einen Blick in den beigefügten \LaTeX{} Header. \\

    Optional lernen wir die Möglichkeit eigens definierter Commands kennen. 
\end{document}