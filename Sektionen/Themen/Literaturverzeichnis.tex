\documentclass{subfiles}

\begin{document}
    \begin{table}[H]
        \centering
        \begin{tabular}{|ccc|}
            \textbf{Folienbereich} & \textbf{Aufgaben} & \textbf{Zeit} \\
            \hline\hline
            \pgfmathparse{\Kapitelseiten[8]}\pgfmathresult & \pgfmathparse{\Aufgaben[8]}\pgfmathresult & \pgfmathparse{\Zeiten[8]}\pgfmathresult
        \end{tabular}
    \end{table}

    Bei dem Schreiben von Hausarbeiten oder Berichten ist es notwendig, verwendete Literatur zu kennzeichnen. Hierzu lernen wir die \LaTeX{} Umgebung \emph{BibTeX} kennen, um Literatur zu verwalten und in einem Literaturverzeichnis aufzulisten. Für die Zitierung in einem Fließtext werden wir den \emph{cite} Befehl kennenlernen. \\

    Optional betrachten wir die Möglichkeit des \emph{manuellen Literaturverzeichnisses} mithilfe von \emph{thebibliography}.
\end{document}