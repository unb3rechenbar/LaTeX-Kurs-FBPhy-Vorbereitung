\documentclass{subfiles}

\begin{document}
    \begin{table}[H]
        \centering
        \begin{tabular}{|ccc|}
            \textbf{Folienbereich} & \textbf{Aufgaben} & \textbf{Zeit} \\
            \hline\hline
            \pgfmathparse{\Kapitelseiten[7]}\pgfmathresult & \pgfmathparse{\Aufgaben[7]}\pgfmathresult & \pgfmathparse{\Zeiten[7]}\pgfmathresult
        \end{tabular}
    \end{table}

    Wir lernen hier konkret das Paket \emph{siunitx} kennen, um physikalische Einheiten zu setzen. Hierzu erlernen wir an einigen Beispielen den \emph{Syntax} und die \emph{Einbindung von Messunsicherheiten} kennen.
\end{document}