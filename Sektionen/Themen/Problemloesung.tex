\documentclass{subfiles}

\begin{document}
    \begin{table}[H]
        \centering
        \begin{tabular}{|ccc|}
            \textbf{Folienbereich} & \textbf{Aufgaben} & \textbf{Zeit} \\
            \hline\hline
            \pgfmathparse{\Kapitelseiten[10]}\pgfmathresult & \pgfmathparse{\Aufgaben[10]}\pgfmathresult & \pgfmathparse{\Zeiten[10]}\pgfmathresult
        \end{tabular}
    \end{table}

    Als unumgänglich in der täglichen \LaTeX{} anwendung stellt sich die Fehlerbehebung heraus. Hierzu schauen wir uns sogenannte \emph{Warnings} und \emph{Errors} an und werfen einen Blick auf eine \emph{Beispielfehlermeldung}. Um bei der Partnerarbeit häufig auftretende Fehler besser zu erkennen oder gar vorzubeugen, bieten wir \emph{Ideen zur Organisation} an.
    
    Optional betrachten wir \emph{externe Fehler} und bieten eine grobe Schablone der \emph{Fehlerbehebung}. Hier weisen wir jedoch explizit auf die Verwendung des Internets hin, da hier die meisten Fehler bereits behandelt wurden. Gute Anlaufstellen sind hier \emph{StackExchange}, \emph{Overleaf} oder Sprachmodelle wie \emph{chatGPT}. 
\end{document}