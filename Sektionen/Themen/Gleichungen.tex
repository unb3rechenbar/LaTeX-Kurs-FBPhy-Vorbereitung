\documentclass{subfiles}

\begin{document}
    \begin{table}[H]
        \centering
        \begin{tabular}{|ccc|}
            \textbf{Folienbereich} & \textbf{Aufgaben} & \textbf{Zeit} \\
            \hline\hline
            \pgfmathparse{\Kapitelseiten[2]}\pgfmathresult & \pgfmathparse{\Aufgaben[2]}\pgfmathresult & \pgfmathparse{\Zeiten[2]}\pgfmathresult
        \end{tabular}
    \end{table}

    In diesem Kapitel beschäftigen wir uns mit einem der Hauptgründe, weshalb sich \LaTeX{} für naturwissenschaftliche Anwendungen eignet: der Formelsatz. Wir lernen die \emph{grundlegenden Umgebungen} hierzu kennen und betrachten \emph{einfache Symbole} und häufig verwendete \emph{griechische Variablennamen} und \emph{mathematische (Funktions)-Namen} kennen.
\end{document}