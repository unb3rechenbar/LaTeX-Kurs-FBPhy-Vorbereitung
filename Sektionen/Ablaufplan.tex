\documentclass{subfiles}

\begin{document}
    In unseren \emph{vier} Terminen müssen wir eine Reihe von Themen besprechen, um einen guten Überblick über das Themenfeld \LaTeX{} zu bekommen. Damit wir dies auch schaffen, haben wir uns einen Ablaufplan überlegt, welcher die Themen auf die Termine verteilt. Da wir jeden Termin nur insgesamt zwei Stunden Zeit haben werden, sind die einzelnen Kapitel durch maximal $60$ Minuten nach oben beschränkt, und dies auch nur bei maximal drei Themen in einer Sitzung. Unser Ablaufplan sieht daher wie folgt aus:
    \begin{table}[H]
        \centering
        \begin{tabular}{c|lr|c|c}
            \textbf{Termin} & \textbf{Thema} & \textbf{Referent} & \textbf{Zeitaufwand} & \textbf{Folienbereich}\\
            \hline\hline
            07.11.2023 & Konzeptvorstellung &  & \pgfmathparse{\Zeiten[0]}\pgfmathresult & \pgfmathparse{\Kapitelseiten[0]}\pgfmathresult \\
             & Design &  & \pgfmathparse{\Zeiten[1]}\pgfmathresult & \pgfmathparse{\Kapitelseiten[1]}\pgfmathresult \\
             & Gleichungen & & \pgfmathparse{\Zeiten[2]}\pgfmathresult & \pgfmathparse{\Kapitelseiten[2]}\pgfmathresult \\
            \hline\hline
            14.11.2023 & Grafiken & & \pgfmathparse{\Zeiten[3]}\pgfmathresult & \pgfmathparse{\Kapitelseiten[3]}\pgfmathresult \\
             & Tabellen & & \pgfmathparse{\Zeiten[4]}\pgfmathresult & \pgfmathparse{\Kapitelseiten[4]}\pgfmathresult \\
             & Positionierung & & \pgfmathparse{\Zeiten[5]}\pgfmathresult & \pgfmathparse{\Kapitelseiten[5]}\pgfmathresult \\
             & Listen & & \pgfmathparse{\Zeiten[6]}\pgfmathresult & \pgfmathparse{\Kapitelseiten[6]}\pgfmathresult \\
            \hline\hline
            21.11.2023 & Physikalische Einheiten & & \pgfmathparse{\Zeiten[7]}\pgfmathresult & \pgfmathparse{\Kapitelseiten[7]}\pgfmathresult \\
             & Literaturverzeichnis & & \pgfmathparse{\Zeiten[8]}\pgfmathresult & \pgfmathparse{\Kapitelseiten[8]}\pgfmathresult \\
             & Titel, Header und Pakete & & \pgfmathparse{\Zeiten[9]}\pgfmathresult & \pgfmathparse{\Kapitelseiten[9]}\pgfmathresult \\
            \hline\hline
            28.11.2023 & Problemlösung & & \pgfmathparse{\Zeiten[10]}\pgfmathresult & \pgfmathparse{\Kapitelseiten[10]}\pgfmathresult \\
             & Ausblick & & \pgfmathparse{\Zeiten[11]}\pgfmathresult & \pgfmathparse{\Kapitelseiten[11]}\pgfmathresult \\
        \end{tabular}
        \caption{Verteilung des Inhaltes auf vier Termine.}
    \end{table}
\end{document}