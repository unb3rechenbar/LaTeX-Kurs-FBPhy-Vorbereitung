% Beginn des Headers mit documentclass
\documentclass[bibliography=totocnumbered,ngerman]{scrartcl}
% scrartcl - Artikelformat der KOMA-Klasse
% bibliography=totocnumbered - Literaturverzeichnis wird in das Inhaltsverzeichnis aufgenommen
% ngerman - Für das Dokument wird die Sprache "Deutsch" eingestellt. Damit wird u.a. in SIUnitX die Bereichsangabe mit "bis" statt "to" geschrieben.

% Direkte Angabe von Umlauten im Dokument
\usepackage[utf8]{inputenc}

% Deutsche Sprachanpassungen
\usepackage[ngerman]{babel}

% Erweiterung des Mathe-Satzes (u.a. align-Umgebungen)
\usepackage{amsmath}

% zusätzliche mathematische Symbole (enthält amssymb)
\usepackage{amsfonts}

% Einbinden von Bildern
\usepackage{graphicx}

% Positionierung von Bildern, Tabellen, ...
\usepackage{float}

% Paket für SI-Einheiten
\usepackage{siunitx}
\sisetup{locale = DE}

% Literaturverzeichnis

%% Layout des Literaturverzeichnisses
\usepackage[autostyle]{csquotes}
\bibliographystyle{ieeetr}


% ALTERNATIVE ZU biblatex:
\usepackage[german]{babelbib}

% Zum Einfügen von URLs (hier benötigt fürs Literaturverzeichnis in Zsh mit babelbib)
\usepackage{url}


% Neuer definierter Befehl für Kapitälchen in Überschriften
\newcommand{\textSC}[1]{{\normalfont  \textsc{#1}}}
